\chapter{Glossary}
%
\begin{description}
%
\item[3D printer]
%
\item[afford] a possibility for use we are able to recognize in an object; our recognition of these possibilities structures our perception of the environment \citep{gibson_1979}
%
\item[affordance] a use afforded by an object (see {\bf afford})
%
\item[Arduino]
%
\item[artifact ecology]
%
\item[\emph{avatar}]
%
\item[\emph{bench}]
%
\item[\emph{bunkle}]
%
\item[\emph{clayticle}]
%
\item[claytronic]
%
\item[computer numerically controlled (CNC) mill]
%
\item[\emph{crystal}]
%
\item[cyborg]
%
\item[direct manipulation] introduced by \citet{shneiderman_computer1983} to describe onscreen interfaces that allowed the mouse to grab and manipulate things; applied here to describe tangible interfaces that allow your hand to grab and manipulate things
%
\item[\emph{duck}]
%
\item[\emph{duckcicle}]
%
\item[ensemble] a group of objects that coordinate to produce a global behavior (especially a group of robotic modules)
%
\item[ensemble of robotic modules] (see {\bf ensemble}, {\bf robot})
%
\item[\emph{fig}]
%
\item[fused deposition modeling (FDM)]
%
\item[\emph{golem}] a robotic agent capable of performing useful manual tasks; from the fictional golem, an agent formed out of clay to serve and protect its creator
%
\item[google] (verb)
%
\item[graphical user interface (GUI)]
%
\item[hackerspace]
%
\item[homunculus] (plural: {\bf homunculi}) an anthropomorphization of a cognitive process or the soul as a tiny person, often depicted as sitting inside someone's head; from the latin diminutive of man (homo)
%
\item[\emph{hyperduck}]
%
\item[\emph{hyperfig}]
%
\item[\emph{hyperform}] a form that varies over time; from `hyper-' indicating extent into the fourth dimension as in hypercube
%
\item[\emph{hyperthou}]
%
\item[\emph{idol}]
%
\item[\emph{leveraged environment}]
%
\item[maker] a person versed in the skills necessary to craft objects
%
\item[mecha] a robot with a human driver inside, a common theme in Japanese fiction
%
\item[manipulative] an object or kit of objects intended to engage children in discovering a particular concept or group of concepts through play
%
\item[manipulative morphology] one of the classes of forms developed to serve as manipulatives (see {\bf morphology})
%
\item[modular robot] a {\bf robot} intended to serve as a member of an {\bf ensemble} of physically coupled modules
%
\item[morphology] the structure and configuration of an object
%
\item[\emph{mount}]
%
\item[open hardware]
%
\item[\emph{organelle}]
%
\item[pick-and-place]
%
\item[\emph{popsicle}]
%
\item[purpose] the broad use case an {\bf ensemble of robotic modules} is intended to support
%
\item[quick response (QR) code] a black-and-white printed pattern designed to be interpreted by a device with a digital camera such as a cell phone
%
\item[radio-frequency identification (RFID) tag] a small passively powered radio transmitter---when it passes near to a tag reader it captures power from the radio signal emitted from the reader and responds with a unique number (and sometimes additional stored data)
%
\item[rapid prototyping]
%
\item[responsive environment]
%
\item[robot] a device capable of sensing, planning and acting
%
\item[\emph{robunculi}] the modules of a {\bf robunculi kit}; from {\bf robot} and {\bf homunculi}; these `little robots' extend our agency out into the environment by allowing us to impress behaviors upon the objects we construct out of them
%
\item[scaffold] (verb) to put a student in a position in which they are able to make discoveries
%
\item[self-reconfiguring materials]
%
\item[sensorimotor] sensory integration for example between the hand and the eye
%
\item[\emph{shrine}]
%
\item[smart phone]
%
\item[\emph{sockpuppet}]
%
\item[\emph{spoke}]
%
\item[\emph{stick}]
%
\item[\emph{stickboard}]
%
\item[\emph{stickpuppet}]
%
\item[tangible interface] a physical interface to digital information
%
\item[tangible sketch] a model built from a kit that communicates spatial information to a computational agent
%
\item[\emph{taster}]
%
\item[\emph{tek}]
%
\item[\emph{temple}]
%
\item[\emph{theater}]
%
\item[\emph{thou}]
%
\item[\emph{tink}]
%
\item[\emph{tinker}]
%
\item[\emph{tinkit}]
%
\item[\emph{tooler}]
%
\item[\emph{volticle}]
%
\item[\emph{wild environment}]
%
\item[\emph{wiz}]
%
\item[\emph{wrangler}]
%
\end{description}

