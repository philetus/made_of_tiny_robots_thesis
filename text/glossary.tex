\chapter{Glossary}
%
\begin{description}
%
\item[afford] a possibility for use we are able to recognize in an object; our recognition of these possibilities structures our perception of the environment \citep{gibson_1979}
%
\item[affordance] a use afforded by an object (see {\bf afford})
%
\item[direct manipulation] introduced by \citet{shneiderman_computer1983} to describe onscreen interfaces that allowed the mouse to grab and manipulate things; applied here to describe tangible interfaces that allow your hand to grab and manipulate things
%
\item[ensemble] a group of objects that coordinate to produce a global behavior (especially a group of robotic modules)
%
\item[ensemble of robotic modules] (see {\bf ensemble}, {\bf robot})
%
\item[golem] an agent capable of performing useful tasks; from the fictional golem, an agent formed out of clay to serve and protect its creator
%
\item[homunculus] (plural: {\bf homunculi}) an anthropomorphization of a cognitive process or the soul as a tiny person, often depicted as sitting inside someone's head; from the latin diminutive of man (homo)
%
\item[hyperform] a form that varies over time; from `hyper-' indicating extent into the fourth dimension as in hypercube
%
\item[maker] a person versed in the skills necessary to craft objects
%
\item[manipulative] an object or kit of objects intended to engage children in discovering a particular concept or group of concepts through play
%
\item[manipulative morphology] one of the classes of forms developed to serve as manipulatives (see {\bf morphology})
%
\item[modular robot] a {\bf robot} intended to serve as a member of an {\bf ensemble} of physically coupled modules
%
\item[morphology] the structure and configuration of an object
%
\item[purpose] the broad use case an {\bf ensemble of robotic modules} is intended to support
%
\item[robot] a device capable of sensing, planning and acting
%
\item[robunculi] the modules of a {\bf robunculi kit}; from {\bf robot} and {\bf homunculi}; these `little robots' extend our agency out into the environment by allowing us to impress behaviors upon the objects we construct out of them
%
\item[scaffold] (verb) to put a student in a position in which they are able to make discoveries
%
\item[sensorimotor] sensory integration for example between the hand and the eye
%
\item[tangible interface] a physical interface to digital information
%
\item[tangible sketch] a model built from a kit that communicates spatial information to a computational agent
%
\end{description}
