\section{Introduction}
%
What follows is an investigation into the ways the burgeoning ubiquity of computation will reshape the way we relate to our physical environment.
Putting computers and screens and sensors and networking and even motors in objects whose utility was previously primarily conferred by their physical form will give the objects in our environment internal states and behaviors; our physical environment will become responsive.
In the extreme case of objects composed of {\bf self-reconfiguring materials}%
\footnote{Terms rendered in bold on first use are defined in the Glossary.}
the form of an object will become just another aspect of its behavior.
This change from form-governed objects (where an object's utility is conferred by its form) to behavior-governed objects (where an object's utility is conferred by its behavior) is the hallmark of a {\bf responsive environment}.
%%% - follow up -> wild vs leveraged vs responsive environments
Thus in a responsive environment our role shifts from being the sole actors who make use of the tools we surround ourselves with to being (perhaps leading) members of a social network of robotic%
\footnote{We will be considering robots in the broadest possible sense of any machine capable of sensing input and responding with various behaviors.}
devices.
We suggest that taking an ecological perspective is important to understanding the shift to responsive environments for two reasons:
\begin{enumerate}
\item these networked robotic devices will serve as cybernetic extensions of our capabilities, and the level of the interface will be ecological; and
\item the complexity of producing these devices combined with their intimate connection with and knowledge of our personal lives will change the dynamics of their production, distribution and use.
\end{enumerate}

In the rest of the introduction below we will further elaborate on the evolution of our designed environment, how we will relate to it, and the roles we will adopt in creating and relating to these new robotic artifacts. We will then lay out the project of this document, which is briefly the development of an ontology of responsive environments and the consideration of possible ecologies of their production and use.

\subsection{The Evolution of Our Designed Environment}
%

\subsection{Where do we plug in?}
%
- responsive environments the next stage of cybernetics
- question is not where do we plug the robots into our brains, but where do we plug ourselves into a network of robotic devices (gibson, will i have a chip in my brain?)
- how are we organized, and how can understanding that inform the design of our tools?

\subsection{Whose side is this table on?}
%


For example, when we put touchscreens in the tabletops at a coffee shop, we 
\begin{itemize}
\item Are responsive tables created by designers or programmers or roboticists or someone else?
\item How do I customize the behavior of a responsive table?
\item Who and what is a responsive table talking to?
\item Whose side is a responsive table on?
\end{itemize}

\subsection{An investigation of the Ecology of Responsive Environments}
%
How we can live amongst computers in a way that empowers us to take control of the sorts of environments that their ubiquity will soon allow?  
We begin with a brief survey of some (speculative) ideas about how our minds are organized; we leverage these ideas to outline a model of the {\bf design roles} we can adopt in relating to responsive physical environments.
We then detail the interface that defines each role, and suggest how designers\footnote{'Makers' would be an equally suitable term; we suggest that in the context of responsive environments it will in any case be useful to reexamine the distinction between making and designing} adopting these roles roles would interact with each other to support a {\bf design ecosystem}.
These roles, their interrelations and their respective interfaces together compose our standard interface for responsive environments.

We then justify our initial speculation about minds and roles by demonstrating how applying our standard interface can:

\begin{enumerate}
\item help to manage the complexity of creating responsive environments;
\item provide a basis for evaluating the relative success of techniques, tools and components;
\item support compartmentalization and reuse of successful designs; and
\item suggest new directions for the development of responsive environments.
\end{enumerate}

We will demonstrate this through
- ontology of responsive environments
- case studies of our projects and others



As the philosopher of mind Daniel Dennett is fond of saying, ``\emph{Yes we have a soul; but it's made of lots of tiny robots.}'' \citeyearpar[p. 1]{freedom_evolves}
% 
***INTRO {\bf GOES} HERE*** \citep{society_of_mind}


