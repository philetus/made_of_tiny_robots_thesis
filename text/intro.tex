\section{Introduction}
%
What follows is a proposal for how we%
    \footnote{Who are we? Currently we are people, but we\saveFootnoteCounterStartWithAlph\footnotemark{} would not wish to exclude sufficiently advanced {\bf artificial intelligences} from participating.}%
        \footnotetext{Sometimes we is just myself, the author, and you, the reader.}%
        \restoreFootnoteCounter{}
can live amongst computers in a way that empowers us to take control of the sorts of environments that their ubiquity will soon allow. 
We begin with a brief survey of some (speculative) ideas about how our minds are organized; we leverage these ideas to outline a model of the {\bf design roles}%
\footnote{Terms rendered in bold on first use are defined in the Glossary.}
we can adopt in relating to our physical environment.
We then detail the interface that defines each role, and suggest how designers\footnote{'Makers' would be an equally suitable term; we suggest that in the context of responsive environments it will in any case be useful to reexamine the distinction between making and designing} adopting these roles roles would interact with each other to support a {\bf design ecosystem}.
These roles, their interrelations and their respective interfaces together compose our standard interface for responsive environments.

We then justify our initial speculation about minds and roles by demonstrating how applying our standard interface can:

1) help to manage the complexity of creating responsive environments;
2) provide a basis for evaluating the relative success of techniques, tools and components;
3) support compartmentalization and reuse of successful designs; and
4) suggest new directions for the development of responsive environments.

We will demonstrate this through
- ontology of responsive environments
- case studies of our projects and others

\subsection{Where do we plug this in?}
%
- responsive environments the next stage of cybernetics
- question is not where do we plug the robots into our brains, but where do we plug ourselves into a network of robotic devices (gibson, will i have a chip in my brain?)
- how are we organized, and how can understanding that inform the design of our tools?

As the philosopher of mind Daniel Dennett is fond of saying, ``\emph{Yes we have a soul; but it's made of lots of tiny robots.}'' \citeyearpar[p. 1]{freedom_evolves}
% 
***INTRO {\bf GOES} HERE*** \citep{society_of_mind}


