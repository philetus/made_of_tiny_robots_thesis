%%% Made of Tiny Robots
%%% An Investigation of the Ecology of Responsive Environments
%%%
\section{Introduction}
%
What follows is an investigation into the ways the burgeoning \textbf{ubiquity of computation}%
\footnote{Terms rendered in bold on first use are defined in the Glossary.}
will reshape the way we relate to our physical environment.
Putting computers and screens and sensors and networking and even motors in artifacts whose utility was previously primarily conferred by their physical form will give the artifacts in our environment internal states and behaviors; our physical environment will become responsive.
In the extreme case of artifacts composed of \textbf{self-reconfiguring materials} the form of an artifact will become just another aspect of its behavior.
This change from form-governed artifacts (where an artifacts's utility is conferred by its form) to behavior-governed artifacts (where an artifacts's utility is conferred by its behavior) is the hallmark of a \emph{responsive environment}%
\footnote{Terms coined here are emphasized on first use.}%
.
Thus in a responsive environment our role shifts from being the sole actors who make use of the tools we surround ourselves with to being (perhaps leading) members of a social network of robotic%
\footnote{We will be considering robots in the broadest possible sense of any machine capable of sensing input and responding with various behaviors.}
artifacts.

We suggest that taking an ecological perspective is important to understanding the shift to responsive environments for two reasons:
\begin{enumerate}
\item these networked robotic artifacts will serve as cybernetic extensions of our capabilities, and the level of the interface will be ecological; and
\item the complexity of producing these artifacts combined with their intimate connection with and knowledge of our personal lives will change the dynamics of their production, distribution and use.
\end{enumerate}
To be clear, when we speak of ecology we do not intend it in the sense of the natural environment. We will be investigating \emph{artifact ecologies}, the systems we apply and roles we adopt in the creation and use of physical artifacts, and the network of relations thus engendered.

In the rest of the introduction we will further elaborate on the evolution of artifact ecologies, how we will interface with responsive environments composed of robotic artifacts, and the roles we will adopt in creating and customizing these new robotic artifacts. We will then lay out the project at hand, which is briefly the development of an ontology of responsive environments and the consideration of possible ecologies of their production and use.

\subsection{The Evolution of Artifact Ecologies}
%
-affordance
-found environment
-leveraged environment
-responsive environment

\subsection{Where Do We Plug In?}
%
- responsive environments the next stage of cybernetics
- question is not where do we plug the robots into our brains, but where do we plug ourselves into a network of robotic devices (gibson, will i have a chip in my brain?)
- how are we organized, and how can understanding that inform the design of our tools?
-manipulatives

As the philosopher of mind Daniel Dennett is fond of saying, ``Yes we have a soul; but it's made of lots of tiny robots.'' \citeyearpar[p. 1]{freedom_evolves}

coin robunculi --- the artifacts of which responsive environments are comprised

\subsection{Whose side is this table on?}
%
- artifact ecology

For example, when we put touchscreens in the tabletops at a coffee shop, we 
\begin{itemize}
\item Are responsive tables created by designers or programmers or roboticists or someone else?
\item How do I customize the behavior of a responsive table?
\item Who and what is a responsive table talking to?
\item Whose side is a responsive table on?
\end{itemize}

\subsection{Our Project}
%
The work presented here is motivated by the following thesis:

\begin{em}
our current artifact ecologies will not survive the transition from leveraged to responsive environments.
\end{em}

The question at hand then is: how we can live amongst robots in a way that empowers us to take control of the sorts of environments that ubiquitous computation will soon allow? To address this question we first need to develop a better understanding of the landscape revealed by these technologies. Our project is:

\begin{em}
to develop an ontology of the various roles we may adopt in creating and using robunculi and the methods of relating that typify each of these roles; and to demonstrate the utility of this ontology in the development and evaluation of robunculi, as well as in considering the character of the artifact ecologies engendered by these systems.
\end{em}

In the following section we present an ontology of responsive environments describing the roles we may adopt in relating to robunculi, the interrelations between these roles and the methods of relation that typify each role. Next we will survey projects from a variety of fields incliding tangible interfaces and modular robotics that satisfy our definition of robunculi and describe them in terms of our ontology. We will then perform case studies of several projects developed by the author and others in further detail to demonstrate how this ontology can assist in identifying reusable techniques and components, and in comparing the relative merits of different projects. In the final section we will draw on these case studies to characterize different possible artifact ecologies and discuss their social implications.




