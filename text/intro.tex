%%% Made of Tiny Robots
%%% An Investigation of the Ecology of Responsive Environments
%%%
\chapter{Introduction}
\label{ch:intro}
%
What follows is an investigation into the ways the burgeoning \textbf{ubiquity of computation}%
\footnote{Terms rendered in bold on first use are defined in the Glossary.}
will reshape the way we relate to our physical environment.
Putting computers and screens and sensors and networking and even motors in artifacts whose utility was previously primarily conferred by their physical form will give the artifacts in our environment internal states and behaviors; our physical environment will become responsive.
In the extreme case of artifacts composed of \textbf{self-reconfiguring materials} the form of an artifact will become just another aspect of its behavior.
This change from form-governed artifacts (where an artifacts's utility is conferred by its form) to behavior-governed artifacts (where an artifacts's utility is conferred by its behavior) is the hallmark of a \emph{responsive environment}%
\footnote{Terms coined here are emphasized on first use.}%
.
Thus in a responsive environment our role shifts from being the sole actors who make use of the tools we surround ourselves with to being (perhaps leading) members of a social network of robotic%
\footnote{We will be considering robots in the broadest possible sense of any machine capable of sensing input and responding with various behaviors.}
artifacts.

We suggest that taking an ecological perspective is important to understanding the shift to responsive environments for two reasons:
\begin{enumerate}
\item these networked robotic artifacts will serve as cybernetic extensions of our capabilities, and the level of the interface will be ecological; and
\item the complexity of producing these artifacts combined with their intimate connection with and knowledge of our personal lives will change the dynamics of their production, distribution and use.
\end{enumerate}
To be clear, when we speak of ecology we do not intend it in the sense of the natural environment.
We will be investigating \bf{artifact ecologies}, the systems we apply and roles we adopt in the creation and use of physical artifacts, and the network of relations thus engendered.

In the rest of the introduction we will further elaborate on the evolution of artifact ecologies, how we will interface with responsive environments composed of robotic artifacts, and the roles we will adopt in creating and customizing these new robotic artifacts.
We will then lay out the project at hand, which is briefly the development of an ontology of responsive environments and the consideration of possible ecologies of their production and use.

\section{The Evolution of Artifact Ecologies}
%
To understand the manner in which we are embedded in ecologies of the creation, distribution and use of artifacts we need a theory of how we relate to individual artifacts.
Gibson \citeyearpar{gibson_1979} suggests that we are able to appreciate the value of physical artifacts through our (innate) recognition of the uses \textbf{afforded} by their form.
We recognize without effort that a chair affords the possibility of sitting and a door in a wall affords the possibility of entering a building. 

We propose that the history of artifact ecologies consists of two broad eras, and we are now on the cusp of a third era.

As our (primate) ancestors developed the faculties to recognize that, for example, flint could be knapped to produce a blade, which could then be used to skin animals to create clothing, we entered the first era of the \emph{wild environment}. 
These sorts of artifact ecologies are characterized by a distinction between a local collection of useful artifacts with clear affordances, generally carried on one's person, and a vast external environment that does not particularly respond to the human scale and within which affordances can be perceived only with significant knowledge and effort.
Artifacts are generally crafted by the bearer for personal use from raw materials harvested from the wild surroundings. 
The chief social aspect of these artifact ecologies was the verbal transfer of the relevant methods to close acquaintances.

The next era was ushered in by the development of urban settlements. 
In these \emph{leveraged environments} our artifact ecology has expanded to the horizon; we are now surrounded by manufactured artifacts designed to present a variety of helpful affordances.
These artifacts give us leverage over our local conditions, for example doors allow us to easily restrict access to a space.
Artifacts are no longer simply controlled by their bearer and encode complex social dynamics; roads and sidewalks are shared by citizens, stores provide artifacts in exchange for money, doors open for those who bear their keys.
In order to produce these new artifacts in sufficient quantity to literally pave over the natural environment significant social changes are required; artifacts are designed by one set of specialists, produced by another, and distributed by yet another.
As desirable artifacts are mass-manufactured through a complex bureacracy they are frequently unevenly distributed.
The artifact ecologies of this era frequently engender social unrest due to inequities in the control of public artifacts and the distribution of private artifacts.

Many people already carry small computational devices with them and routinely interact with these newly familar robitic interfaces. As this computational ubiquity spreads we are already transitioning into a third era of artifact ecologies: responsive environments. These ecologies will be characterized by the promotion of our artifacts from passive tools to networked social peers. The intimacy of these relationships will place much more power over our behaviors and emotions in the hands of those who dictate the behavior of these artifacts. At the same time these artifacts have the potential to allow mass customization to the desires of those bearing a given artifact. 

\section{Where Do We Plug In?}
%
Science fiction stories have led many people to ask: (when) will we have computer chips in our heads?%
\footnote{I am indebted to William Gibson for this formulation. (cite?)} 
The answer for most of us is probably never, as we do not actually need to cut into our brains to become \textbf{cyborgs}; we are fully capable of interfacing with computers through language, through \textbf{GUI interfaces} and through \textbf{tangible interfaces}.
By embedding ourselves in artifact ecologies populated with robotic devices with these sorts of interfaces, we in effect incorporate these other computational systems into our own thought processes.
The best current example of this kind of cybernetic interface is \textbf{googling}; once one learns basic techniques for interacting with internet search engines and acquires a persistent network interface (such as a \textbf{smart phone}%
\footnote{While this term is currently well known we expect that in the near future it will seem as antiquated as ``personal digital assistant''.}
) one becomes a sort of information-retrieval cyborg.
The googlebot is so easily incorporated into our minds because our minds are already just a collection of specialized computational units that in concert to form a ``society of mind'' \citep{society_of_mind}. 
As the philosopher of mind Daniel Dennett is fond of saying, ``Yes we have a soul; but it's made of lots of tiny robots'' \citeyearpar[][p. 1]{freedom_evolves}. Although these tiny robots have historically happened to all be in our brains, with the advent of responsive environments we will be incorporating more and more robotic devices into our local cybernetic artifact ecologies.

Many of these devices constantly feed data to apparently discorporate agents like the googlebot%
\footnote{The googlebot actually has a physical stature in line with its apparent omniscience; it fills several enormous buildings spread around the world and consumes enormous amounts of energy from both the grid and dedicated power plants.}.
As we enter into a cybernetic relationship with this new artifact ecology much of what we have until now considered our private personas will be determined by the behavior of the robotic devices we surround ourselves with, and by our relationship with the computational agents that manage these devices and mediate our interactions with other people. 
In light of our special relationship we will refer to these agents (like the googlebot) generally as \emph{idols}%
\footnote{We chose the name `idol' after Gibson's `idoru'\citeyearpar{gibson_idoru}, a literal AI rock star, with the accompanying cultural leverage of a teen idol; and after religious idols, to draw an analogy with the religious practice of asking powerful ethereal agents for advice and favors.}.

\section{Potential Responsive Environment Ecologies}
As we move toward responsive environments several potentially stable ecological configurations are emerging. 
The character of our artifact ecology, and what it means to be and individual within this new order, will depend largely on three factors: the transparency of behaviors, the reconfigurability of devices, and the consolidation of production.



As we are getting so familiar with this sort of robotic device we propose to give them a name: \emph{robunculi}%
\footnote{Applying the Latin plural diminutive `-unculi' to `robot' gives `robunculi', literally `little robots'.}.
This name is a play on \textbf{homunculus}, an anthropomorphization of the human soul visualized as a little man sitting in our head issuing instructions, or sometimes as two little men on our shoulders whispering arguments into our ears.
(TODO: figure)
With the advent of robunculi we now have the googlebot whispering to us through our smartphones.

While our leveraged environment has familiarized us with the affordances of a wide variety of specific forms (i.e. cups and bowls, hammers and nails, shoelaces and zippers) as the physical artifacts in our environment gain internal states and behaviors we will need new models for recognizing and exploiting the affordances presented by robunculi. As models we propose to adopt \textbf{manipulatives}, tangible interfaces, \textbf{modular robots} and \textbf{self-reconfiguring materials}, as we will discuss further as we develop an ontology of responsive environments in the following section.

(TODO: expand, maybe whole introductory section on these models?)

\section{Our Project}
%
The work presented here is motivated by the following thesis:
\begin{em}
our current artifact ecologies will not survive the transition from leveraged to responsive environments.
\end{em}

(TODO: expand)

The question at hand then is: how we can live amongst robots in a way that empowers us to take control of the sorts of environments that ubiquitous computation will soon allow? To address this question we first need to develop a better understanding of the landscape revealed by these technologies. Our project is:
\begin{em}
to develop an ontology of the various roles we may adopt in creating and using robunculi and the methods of relating that typify each of these roles; and to demonstrate the utility of this ontology in the development and evaluation of robunculi, as well as in considering the character of the artifact ecologies engendered by these systems.
\end{em}

In the following section we present an ontology of responsive environments describing the roles we may adopt in relating to robunculi, the interrelations between these roles and the methods of relation that typify each role. Next we will survey projects from a variety of fields incliding tangible interfaces and modular robotics that satisfy our definition of robunculi and describe them in terms of our ontology. We will then perform case studies of several projects developed by the author and others in further detail to demonstrate how this ontology can assist in identifying reusable techniques and components, and in comparing the relative merits of different projects. In the final section we will draw on these case studies to characterize different possible artifact ecologies and discuss their social implications.




