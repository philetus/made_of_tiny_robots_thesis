\section{Introduction}
%
What follows is a proposal for a {\bf standard interface}\footnote{Terms rendered in bold on first use are defined in the Glossary.} for the specification and control of {\bf responsive environments}. We begin with a brief survey some (speculative) ideas about how people are organized; we leverage these ideas to outline a model of the {\bf design roles} we can adopt in relating to our physical environment. 

We then justify our initial speculation by demonstrating how applying our standard interface can:

1) help to manage the complexity of creating responsive environments;
2) provide a basis for evaluating the relative success of techniques, tools and components;
3) support compartmentalization and reuse of successful designs; and
4) suggest new directions for the development of responsive environments.

We will demonstrate this through
- ontology of responsive environments
- case studies of our projects and others

\subsection{Where do we plug this in?}
%
- responsive environments the next stage of cybernetics
- question is not where do we plug the robots into our brains, but where do we plug ourselves into a network of robotic devices (gibson, will i have a chip in my brain?)
- how are we organized, and how can understanding that inform the design of our tools?

As the philosopher of mind Daniel Dennett is fond of saying, ``\emph{Yes we have a soul; but it's made of lots of tiny robots.}'' \citeyearpar[p. 1]{freedom_evolves}
% 
***INTRO {\bf GOES} HERE*** \citep{society_of_mind}


