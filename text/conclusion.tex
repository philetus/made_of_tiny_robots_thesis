%%% Made of Tiny Robots
%%% An Investigation of the Ecology of Responsive Environments
%%%
\chapter{Conclusion}
\label{ch:conclusion}
%

\subsection{nodes of power}
        \begin{enumerate}
            \item manufacturing
            \item data transmission
            \item data stores
            \item shrines (high-powered computing clusters)
            \item leaf node control
        \end{enumerate}
        
        
\section{Potential (Artifact) Ecological Impacts of Responsive Environments}
\label{sec:ecological_impacts}
%
\subsection{Idols}
%
While we have related to our leveraged environments primarily as tool-users, as our environment becomes responsive this relationship is becoming more of a partnership amongst ourselves and various computational agents. 
For example in a leveraged environment professionals would often supplement their memory by using a notebook and appointment book, and supplement their expertise with a small private library. 
In a responsive environment, professionals supplement their memory through a small computer carried on their person (which we will call a \emph{crystal}\footnote{We call this class of handheld computers that provide a constant connection to the network `crystals', as in a crystal ball that is used to view distant places and communicate with other agents through the ether, and as a reference to their current popular physical realization as a fragile glass-plated touchscreen.}) that manages notes and appointments through the interaction of a variety of software agents over a network; they supplement their expertise by using this same device to query an idol (i.e. the googlebot).
Of course professionals in a leveraged environment also practice division of labor, with secretaries managing appointments and engineers on call to make judgements using their personal libraries to supplement their domain knowledge.
The difference is that in a leveraged environment partnerships are between people, and artifacts only effect change when operated by a person.
Now when people send us invitations (from their crystals, over the network) an idol helpfully inserts them into our calendars and then buzzes us through our own crystals at the appropriate time to tell us where to be.

We suggest that the outlook for accepting these idols into our cybernetic consciousness varies from the utopian to the orwellian largeley depending on: how much control we (the citizens of idol-mediated societies) have over the behavior of our devices and idols; and how transparent (to us) the mechanisms governing these behaviors are. But before we can discuss the qualities of these computational artifacts we should discuss what kinds of artifacts we expect to see.

    \begin{enumerate}
        \item radical transparency - big brother and little brother
        \item means of production 2 - factories vs 3d printers
        \item battle of the heavens - corporate clouds vs govt clouds vs community clouds
        \item digital serfdom and device transparency
    \end{enumerate}
    
    [TODO: figures showing axes] (transparency - physically secured state/corp rental -> drm'd black box -> open source hardware) (reconfigurability - mass-manufactured widget -> bespoke popsicle -> kit of parts -> hyperform)
