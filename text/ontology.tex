%%% Made of Tiny Robots
%%% An Investigation of the Ecology of Responsive Environments
%%%
\chapter{Robunculi: an Ontology and Ecological Analysis}
%
As we enter the age of ubiquitous computing \citep{weiser_1999} our relationship with our artifact ecology is changing. 
There are several factors to these changes: we have new ways to fabricate artifacts; we have new ways of relating to artifacts; and our artifacts increasingly perform tasks that were once the exclusive domain of people.
One of the key changes in an artifact ecology (that drives other changes) is the emergence of powerful computational agents that manage the explosion of relationships (and data) that arise when every artifact begins to communicate over the network.
All of these artifacts feed into online data stores physically situated in massive computing clusters (which we will refer to generally as \emph{temples}), managed by agents like the googlebot (which we will refer to generally as \emph{idols}\footnote{The names temple and idol are chosen to draw an analogy with the way earlier artifact cultures attempt to curry favor from, and ask the advice of, powerful ephemeral agents}).
As we enter into a cybernetic relationship with this new artifact ecology much of what we have until now considered our private personas will be determined by the behavior of the robotic devices we surround ourselves with, and by our relationship with the idols that manage these devices and mediate our interactions with others.
The character of this ecology, and what it means to be and individual within this new order, will depend largely on two factors: the transparency of these mechanisms to individuals, and their reconfigurability.

[TODO: walk through rest of chapter]

\section{Idols}
%
While we have related to our leveraged environments primarily as tool-users, as our environment becomes responsive this relationship is becoming more of a partnership amongst ourselves and various computational agents. 
For example in a leveraged environment professionals would often supplement their memory by using a notebook and appointment book, and supplement their expertise with a small private library. 
In a responsive environment, professionals supplement their memory through a small computer carried on their person (which we will call a \emph{crystal}\footnote{We call this class of handheld computers that provide a constant connection to the network 'crystals', as in a crystal ball that is used to view distant places and communicate with other agents through the ether, and as a reference to their current popular physical realization as a fragile glass-plated touchscreen.}) that manages notes and appointments through the interaction of a variety of software agents over a network; they supplement their expertise by using this same device to query an idol (i.e. the googlebot).
Of course professionals in a leveraged environment also practice division of labor, with secretaries managing appointments and engineers on call to make judgements using their personal libraries to supplement their domain knowledge.
The difference is that in a leveraged environment partnerships are between people, and artifacts only effect change when operated by a person.
Now when people send us invitations (from their crystals, over the network) an idol helpfully inserts them into our calendars and then buzzes us through our own crystals at the appropriate time to tell us where to be.

[TODO: tie into theme of transparency and reconfigurability]

\section{Kinds of Artifacts}

\section{Examples of Artifact Ecologies}
        nodes of power
        \begin{enumerate}
            \item manufacturing
            \item data transmission
            \item data stores
            \item shrines (high-powered computing clusters)
            \item leaf node control
        \end{enumerate}        


\section{Robunculi: an Ontology}


Below we outline an ontology of concepts [TODO]

> robunculi are a vision of a stable artifact ecology that is both transparent and reconfigurable. while this vision is extremely speculative, by establishing a language for discussing responsive environments as well as metrics for assessing the desirability for individual participants we can make a valuable contribution


> robunculi - convergence of responsive artifacts and kits of parts

> axes of ubiquity, transparency and reconfigurability

> [TODO: figures showing axes] (transparency - physically secured state/corp rental -> drm'd black box -> open source hardware) (reconfigurability - mass-manufactured widget -> custom-printed widget -> kit of parts -> hyperform)

\begin{enumerate}
    \item an ontology of responsive environments
    \begin{enumerate}
        \item robunculi typologies
        \begin{enumerate}
            \item idols
            \item tangible sketches
            \item golems
            \begin{enumerate}
                \item sock puppet (dumb rc golem)
                \item avatar (golem serving as interface to idol)
            \end{enumerate}
            \item hyperforms
        \end{enumerate}
        \item morphologies
        \begin{enumerate}
            \item tile
            \item block
            \item skeleton (graph)
            \item panel
            \item glass (screen / projection interface)
            \item shrine (idol-scale computing facility)
        \end{enumerate}
        \item affordances
        \begin{enumerate}
            \item parallel affordances are synergistic
            \item placing / self-reconfiguring
            \item posing / flexing (self-posing)
            \item commanding (pointing) / signalling (haloing)
            \item listening (tagging) / responding (texting)
            \item graffing (accepting drawings) / gramming (responding with drawings)
            \item puppeteering / puppeting (present puppeteering interface)
            \item sinks generate structured data to be accessed through idols
            \item logging (recording interactions to data stores) (sink)
            \item crawling (indexing data stores) (sink)
            \item tracking (id-ing and classifying agents with sensors) (sink)
            \item slamming (exploring and mapping environments) (sink)
        \end{enumerate}
    \end{enumerate}
    \item an analysis of the potential (artifact) ecological impacts of responsive environments
    \begin{enumerate}
        \item radical transparency - big brother and little brother
        \item means of production 2 - factories vs 3d printers
        \item battle of the heavens - corporate clouds vs govt clouds vs community clouds
        \item digital serfdom and device transparency
    \end{enumerate}
    \item what things *arent* robunculi?
    \begin{enumerate}
        \item construction kits vs rapid prototyping
        \item robunculi, productization and reuse
    \end{enumerate}
\end{enumerate}
